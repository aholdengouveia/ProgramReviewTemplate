INTRODUCTION - BACKGROUND

REGIONAL ACCREDITATION CONTEXT FOR PROGRAM REVIEW 

NECHE Standard 2.7: The institution’s principal evaluation focus is the quality, integrity, and effectiveness of its academic programs. 

NECHE Standard 4.6: The institution … on a regular cycle reviews its academic programs under institutional policies that are implemented by designated bodies with established channels of communication and control. Review of academic programs includes evidence of student success and program effectiveness and incorporates an external perspective. Faculty have a substantive voice in these matters. 


SCHEDULING OF PROGRAM REVIEWS 

The Office of the Vice President of Academic and Student Affairs shall maintain a copy of the current schedule for programs to be reviewed, including the names of the person(s) designated as program review team leader(s). The schedule shall be developed in consultation with the Deans/ Assistant Deans, and shall be posted on the College’s website. 

FORMATION OF PROGRAM REVIEW TEAM 

The team shall consist of a team leader and 3-5 other team members, as follows: 

A. Faculty 
I. If there are full-time faculty members in the program in addition to the designated team leader, then at least one should be included on the team. 
II. The team leader may include part-time/adjunct faculty on the team.
III. The team leader is encouraged to include on the team, or just seek input from, faculty members outside the department/program/division if necessary.

B. External Representative 
I. At least one Advisory Committee member or other external content expert may be identified as a resource if needed. 
II. Programs are encouraged to present relevant findings from the Program Review to the program’s Advisory Board upon completion of the review.

SUGGESTED TIMELINE FOR PROGRAM REVIEW PREPARATION 

NOTE: The program review calls for the insertion of the program’s Curriculum Map and Core Skills Audit Sheet. These documents will be very helpful is addressing many sections of the program review report.

Date
Activity
Feb. 15 (Year 1)
Inform/Orient: A representative from the Office of Academic and Student Affairs confirms schedule with the Deans and/or Assistant Deans of the programs scheduled for program review (due March 1 of the following year). Deans and/or Assistant Deans designate a team leader to run the program review process.
Feb. 15 (Year 1)
Orientation: All program review leaders, their supervisors, a representative from the Office of the Vice President of Academic \& Student Affairs, and a representative from Institutional Research \& Planning attend an orientation meeting.
Spring/
Summer (Year 1)
Assemble team/Begin meeting: Team leader identifies members of the team and determines which program review template is applicable, depending on accreditation status.
Spring/
Summer (Year 1)
Develop and/or review the program’s mission statement and program outcomes. Team leader may need to communicate with others who are doing assessment work.
Spring/
Summer (Year 1)
Develop, review and/or revise curriculum map.
Fall (Year 1)
Request data: Team leader requests necessary data from IRP (see below); determines what information is necessary to gather from Dean and other faculty.
Fall/Winter (Year 1 and 2)
Complete Program Review document.
Jan. 15 (Year 2)
Submit for Program Coordinator/Department Chair review.
Feb. 15 (Year 2)
Submit for Dean/Assistant Dean/Director review.
March 15 (Year 2)
Submit for Vice President of Academic and Student Affairs review.
April 30 (Year 2)
Presentations at Annual Program Review Summit.
May 1 (Year 2)
Deliverables provided to Vice President of Academic and Student Affairs (electronic copies of program review document and of summit presentation).
May 15 (Year 2)
Program review documents posted to website.
June 1 (Year 3)
Dean/Assistant Dean provides written update on progress on program review action plan to Vice President of Academic and Student Affairs.





INSTRUCTIONS FOR OBTAINING DATA FOR PROGRAM REVIEW

Appendix 1: Assessment Documents

In Appendix 1, you must attach the program’s Core Skills Audit Sheet (Associate’s degree only) and Curriculum Map. Programs are required to keep these documents up to date. Please consult the person(s) responsible for Assessment in your program.


Appendix 2: Charts 1-2

Chart 1: Faculty Resources asks for the names and ranks of all faculty members in your program, and the number of credit hours they taught in the last full academic year. You are responsible for compiling this information, or designating someone in your department to do so.

The team leader should send a blank version of Chart 2: Faculty Credentials to all faculty in the program for faculty to fill out and return.


Appendix 3: Chart 3-5 

The Office of Institutional Research and Planning will provide completed Charts 3-5 (Student Demand; Retention, Graduation and Transfer; and Student Demographics) upon request. During the Fall or Summer term prior to Program Review due date, submit a ticket in the NECC Service Desk ticket system (http://servicedesk.necc.mass.edu) in the IRP queue. In your request, please specify:
    • That you are working on a Program Review and need to be provided the completed Charts 1, 3-5;
    • When the Program Review is due;
    • What program(s), including program code, your Program Review is for.
Please note that IRP will provide data for the past five academic years. For example, if your Program Review is due in April 2017, the data will include AY 2011-2012, 2012-2013, 2013-2014, 2014-2015, and 2015-2016.
Please expect an approximately 3-4 week turnaround on this request.
